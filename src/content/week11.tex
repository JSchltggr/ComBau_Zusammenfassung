%! Author = joels
%! Date = 27/01/2022

\section{Garbage Collection 2}
\textbf{GC braucht Metadaten:}
\begin{itemize}[topsep=0pt]
    \itemsep -0.2em
    \item Pointer Offsets in Objekt
    \SubItem{Via Typ-Deskriptor}
    \item Pointer Offsets in Stack
    \SubItem{Via Method Descriptor und Interpreter-Typen}
    \item Pointers in register
    \SubItem{Bei Interpreter nicht nötig}
\end{itemize}

\subsection{Finalizer}
\begin{itemize}[topsep=0pt]
    \itemsep -0.2em
    \item Eine Methode, die vor Löschen des Objekts läuft
    \SubItem{Abschlussarbeit: Verbindungen schliessen etc.}
    \item Von GC initiiert, wenn Objekt zu Garbage geworden ist
\end{itemize}

\textbf{Finalizer wird nicht in GC-Phase ausgeführt, sondern erst später:}
\begin{itemize}[topsep=0pt]
    \itemsep -0.2em
    \item Finalizer dauert evtl. beliebig lange
    \SubItem{Würde GC blockieren}
    \item Finalizer kann neue Objekte allozieren
    \SubItem{Korrumpiert GC}
    \item Programmierfehler im Finalizer
    \SubItem{Evtl. Crash des GC}
    \item Finalizer kann Objekt wieder weiterleben lassen
    \SubItem{Resurrection}
\end{itemize}

\subsubsection{Resourrection}
\begin{itemize}[topsep=0pt]
    \itemsep -0.2em
    \item Finalizer kann bewirken, dass Objekt wieder lebendig wird, also kein Garbage mehr ist
    \item Nicht nur das eigene Objekt, sondern auch indirekt andere Objekte können wiederaufstehen
\end{itemize}

\subsubsection{Finalizer Internals}
\textbf{Finalizer Set:} Registrierte Finalizer\\
\textbf{Pending Queue:} Noch auszuführende Finalizer\\
Wenn Objekt Garbage wird: Aus Finalizer Set löschen und in Pending Queue hinzufügen. Erst nach Pending Queue definitiv löschen.\\
\textbf{Konsequenzen:}
\begin{itemize}[topsep=0pt]
    \itemsep -0.2em
    \item GC braucht 2 Mark Phasen
    \SubItem{Markiere und erkenne Garbage mit Finalizer}
    \SubItem{Markiere von Pending Queue erneut, dann Sweep}
    \item Speicher kann evtl. nicht schnell genug frei werden
\end{itemize}
\textbf{Programmieraspekte:}
\begin{itemize}[topsep=0pt]
    \itemsep -0.2em
    \item Reihenfolge der Finalizer ist unbestimmt
    \item Finalizer laufen beliebig verzögert
    \item Finalizer sind nebenläufig zum Hauptprogramm
    \SubItem{Seperater Thread oder unbestimmt verzahnt}
    \item Läuft der Finalizer nach resurrection wieder?
    \SubItem{Nicht in java}
\end{itemize}

\subsection{Weak References}
\begin{itemize}[topsep=0pt]
    \itemsep -0.2em
    \item Weak References zählt nicht als referenz für GC
    \item Garbage kann mit Weak Reference noch so lange erreicht werden, bis es abgeräumt worden ist
    \item Weak Reference wird nach Freigabe von Garbage Objekt auf null gesetzt
    \item Weak Reference vor oder nach Finalizer freigeben? $\rightarrow$ Weak, Soft, Phantom Reference
\end{itemize}

\subsection{Compacting GC}
\textbf{Problem:} Grosses Objekt müsste auf Heap alloziert werden, aber passt in keine Lücke (obschon genügend freier Speicher vorhanden wäre)\\
\textbf{Lösung:}
\begin{itemize}[topsep=0pt]
    \itemsep -0.2em
    \item Mark \& Copy GC
    \item Allokation am Heap-Ende
    \item GC schiebt Objekte wieder zusammen
    \item Beim Verschieben müssen alle Referenzen nachgetragen werden
    \item Konservative Methode daher unmöglich
\end{itemize}

\subsection{Inkrementeller GC}
\begin{itemize}[topsep=0pt]
    \itemsep -0.2em
    \item GC soll quasi-parallel zum Mutator laufen
    \item Nur kleinste Unterbrüche (Inkrements)
    \item Kein Stop \& Go / Stop the World $\rightarrow$ Möglichst kurze Unterbrüche
\end{itemize}

\textbf{Inkrementelle GCs}
\begin{itemize}[topsep=0pt]
    \itemsep -0.2em
    \item Generational GC
    \SubItem{Junge Objekte schneller freigeben}
    \SubItem{Junge Objekte: Kurze Lebenserwartung, Alte Objekte: Lange Lebenserwartung}
    \item Partioned GC
    \SubItem{Heap partitionsweise verwalten}
    \item Weitere Verfahren
    \SubItem{Concurrent GC, Real-Time GC}
\end{itemize}

