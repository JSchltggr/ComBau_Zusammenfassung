%! Author = joels
%! Date = 27/01/2022

\section{Code Optimierung}
\textbf{Aufgabe der Optimierung:}
\begin{itemize}[topsep=0pt]
    \itemsep -0.2em
    \item Transformation von Intermediate Representation/Maschinencode zu effizienteren Version
    \item Mögliche Intermediate Representations:
    \SubItem{AST + Symbol Table}
    \SubItem{Bytecode}
    \SubItem{Anderer Intermediater Code (z.B. Three Address Code)}
    \item Meist eine Serie von Optimierungsschritten
\end{itemize}

\subsection{Optimierte Arithmetik}
\textbf{Multiplikation/Division:} Bit Shiften mit der Potenz (mal = links, geteilt = rechts)\\
\textbf{Modulo:} x \% 32 $\rightarrow$ x \& 31

\subsection{Algebraische Vereinfachung}
Expr / 1 $\rightarrow$ Expr\\
Expr * 0 $\rightarrow$ 0\\
- - Expr $\rightarrow$ Expr\\
1 + 3 $\rightarrow$ 4

\subsection{Loop Invariant Code}
\begin{lstlisting}
// Wiederholt ausgewertete Teilausdrücke in temp Variable speichern
while (x < N * M) {
    k = y * M;
    x = x + k;
}

// Optimiert:
k = y * M;
temp = N * M;
while (x < temp) {
    x = x + k;
}
\end{lstlisting}

\subsection{Common Subexpressions}
\begin{lstlisting}
// Wiederholt ausgewertete Teilausdrücke in temp Variable speichern
x = a * b + c
...
y = a * b + d

temp = a * b;
x = temp + c;
...
y = temp + d;
\end{lstlisting}

\subsection{Dead Code}
\begin{lstlisting}
a = readInt();
b = a + 1;
writeInt(a);
c = b / 2; // Nicht gebraucht

// Schritt 1
a = readInt();
b = a + 1; // Nicht mehr gebraucht
writeInt(a);

// Schritt 2
a = readInt();
writeInt(a);
\end{lstlisting}

\subsection{Copy Propagation}
\begin{lstlisting}
// Idee: Ersetze Variable durch zuletzt zugewiesenen Ausdruck
t = x + y;
u = t
writeInt(u);

// Schritt 1
t = x + y;
u = x + y
writeInt(u);

// Schritt 2
t = x + y;
u = x + y
writeInt(x + y);

// Schritt 3
writeInt(x + y);
\end{lstlisting}

\subsection{Constant Propagation}
Auch Constant Folding genannt. $\rightarrow$ Konstante Ausdrücke ersetzen.

\subsection{Partial Redundancy}
Optimieren, dass eine Redundante Berechnung nur ein mal pro Pfad ausgeführt wird.

\subsection{Erkennung von Optimierungspotential}
\begin{itemize}[topsep=0pt]
    \itemsep -0.2em
    \item Static Single Assignment
    \item Peephole Optimization
    \item Dataflow Analysis
\end{itemize}

\subsubsection{Static Single Assignment}
\begin{itemize}[topsep=0pt]
    \itemsep -0.2em
    \item Code-Transformation für einfachere Analyse \& Optimierung
    \item Jedes mal wenn x ein neuer Wert zugewiesen wird, eine neue Variable x2 anlegen
    \SubItem{Phi Function bei Verzweigungen}
    \item Relativ kompliziert und teuer
    \item Günstigere Techniken gewünscht wie z.B. Peephole
\end{itemize}

\subsubsection{Peephole Optimization}
\begin{itemize}[topsep=0pt]
    \itemsep -0.2em
    \item Optimierung für sehr kleine Anzahl Instruktionen
    \item In JIT Compiler für Intermediate Code oder Maschinencode benutzt
    \item Sliding Window mit z.B. 3 Instruktionen
\end{itemize}

\begin{lstlisting}
ldc 2
ldc 1
imul     -->  ldc 2
ldc 4         ldc 4   -->  ldc 6
iadd          iadd
\end{lstlisting}

\subsubsection{Dataflow Analysis}
\begin{itemize}[topsep=0pt]
    \itemsep -0.2em
    \item Mächtige generische Code-Analyse-Technik
    \item Für viele Optimierungen nützlich
\end{itemize}